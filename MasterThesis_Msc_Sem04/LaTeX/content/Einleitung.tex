\chapter{Einleitung}

\section{Motivation}

Die Fähigkeit etwas mit seinen Gedanken zu steuern, ist schon seit langem ein erstrebenswerter Traum der Menschen.
So leicht es in Science-Fiction Büchern erscheinen mag, ist es in der Realität eine nicht zu unterschätzende Herausforderung.
Dennoch handelt es sich hierbei um ein interessantes Themengebiet, das noch am Anfang seiner Entwicklung steht.
Der aktuelle Forschungsstand bedeutet für durchschnittliche Menschen noch nicht viel, doch für einige Menschen mit motorischen Störungen können selbst die kleinsten Entwicklungen in diesem Bereich enorme Möglichkeiten bieten.
Die Forschung der vergangenen Jahrzehnte brachten einige Paradigmen im Bereich der Gehirn-Computer-Schnittstellen hervor.
Seitdem wurde eine Vielzahl von Anwendungen entwickelt, um diesen Menschen eine
Möglichkeit zu geben sich auszudrücken und so ihr Leben zu erleichtern.
Diese Masterarbeit soll in diesem Kontext eine Erweiterung für eine Spiele Engine entwickeln, sodass Erzeugnisse dieser Engine mit eine Gehirn-Computer-Schnittstelle gespielt werden können.\\







\section{Beschreibung der Aufgabenstellung}

Gegenstand dieser Arbeit wird es sein, Konzept und Implementierung einer \acs{BCI}-Erweiterung für eine Point\&\-Click Spiele Engine auszuarbeiten und diese zu realisieren.
Zu diesem Zweck ist es erforderlich, dass mögliche Hard- und Softwarekomponenten untersucht und ausgewählt werden.
Dies erfordert insbesondere die Kopplung der \acs{BCI}-Software und der Spiele Engine in geeigneter Weise.
Des Weiteren wird eine Untersuchung und Auswahl der \acs{BCI}-Paradigmen benötigt, da diese maßgeblichen Einfluss auf die Implementierung der Erweiterung haben.


Abschließend müssen Tests konzipiert und durchgeführt werden, sodass eine Verifizierung der Erweiterung erfolgt und sichergestellt wird, dass die Aufgabe dieser Arbeit korrekt erfüllt wurde.\\




