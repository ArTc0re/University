\chapter*{Zusammenfassung}

Die vorliegende Arbeit befasst sich mit der Realisierung einer \acs{BCI}-Erweiterung für eine Point\&Click Engine.
Es wird der Frage nachgegangen, welche \acs{BCI}-Paradigmen für diesen Zweck geeignet und realisierbar sind.
Aus dieser Frage ergeben sich daher mehrere Möglichkeiten zur Darstellung visueller Stimuli, dessen nähere Auswahl anhand des Point\&Click-Eingabeparadigmas erfolgt.
Zusätzlich werden verschiedene Szenarien für die Präsentation dieser Stimuli betrachtet, sowie deren Vor- und Nachteile spezifiziert.

Die Erstellung des Konzepts erfordert zudem die Recherche und Auswahl einer quell\-offenen Point\&Click Engine, sowie einer ebenso frei verfügbaren BCI-Software für die Verarbeitung und Auswertung der EEG-Daten.
Im Konzept werden die verschiedenen Komponenten detailliert betrachtet und miteinander verknüpft, sowie die Interprozesskommunikation definiert.\\
Während der Implementierung wird infolge des Konzepts ein Plugin für die Point\&\-Click Engine \textit{Adventure Game Studio} und eine Modifikation basierend auf dem P300-Speller des \textit{BCI2000}-Frameworks entwickelt.
Das Plugin erweitert die Engine um zusätzliche Funktionen und ermöglicht die Definition beliebiger Speller-Matrizen.
Beide Komponenten kommunizieren Parameter und Ergebnis und ermöglichen durch abwechselnde Aktivität den Spielfluss.\\
Im Anschluss an die Implementierung wird eigens ein Test-Spiel konzipiert und entwickelt, sowie ein Evaluations-Konzept für die Verifikation der Erweiterung ausgearbeitet.
Die Tests sehen den Vergleich zwischen den Ergebnissen des \textit{BCI2000}-P300-Spellers und des modifizierten P300-Spellers vor und sollen die Korrektheit der Erweiterung verifizieren.
Darüberhinaus beinhalten die User-Tests mehrere Fragebögen und ermitteln die subjektive Belastung des Anwenders mittels NASA TLX.
Abschließend werden die gesammelten Ergebnisse angemessen visualisiert und zeigen, dass die Steuerung eines Point\&Click-Adventures mit der entwickelten \acs{BCI}-Erweiterung möglich ist.\\
