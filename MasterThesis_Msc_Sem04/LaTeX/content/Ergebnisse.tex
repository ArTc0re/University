\chapter{Ergebnisse}

In diesem Kapitel werden die Ergebnisse der Tests vorgestellt und in geeigneter Weise interpretiert.\\
Der gesamte Testablauf dauerte für jede Person etwa eine Stunde. 

Von insgesamt zehn Probanden waren nur sechs in der Lage den Test komplett durchzuführen und verwertbare Ergebnisse zu liefern.
Von den vier ausgeschlossenen Testpersonen erreichten zwei nur eine unzureichende Signalstärke.
Die dritte Person durfte am Test auf Grund von Epilepsie nicht teilnehmen.
Bei der vierten Testperson musste der Test auf Grund von Konzentrationsschwierigkeiten abgebrochen werden.\\


% Definition von Hit, Semi und Miss für den nachfolgenden Kontext

In Bezug auf die Tests und die Ergebnisse ist anzumerken, dass jede Ermittlung eines Zielfelds genau genommen aus der Kombination zweier Eingabe-Ermittlungen hervor geht.
Eine Auswahl setzt sich daher aus der ermittelten Zeile und der ermittelten Spalte zusammen.
Demzufolge müssen für die Interpretation der Ergebnisse zunächst drei Begriffe definiert werden, die ein ermitteltes Ergebnis beschreiben.

 \begin{Def} - \textbf{Hit} \\
 Wir sprechen von einem Hit, wenn eine richtig ermittelte Zeile mit einer richtig ermittelten Spalte kombiniert und somit das korrekte Zielfeld wiedergegeben wird.
 \end{Def}
 \begin{Def} - \textbf{Semi} \\
 Ein Semi tritt immer dann auf, wenn lediglich eine Zeile bzw. Spalte richtig und die andere falsch ermittelt wurde.
 \end{Def} 
 \begin{Def} - \textbf{Miss} \\
Ein Miss entspricht der Kombination aus je einer falsch ermittelten Zeile und Spalte.\\
 \end{Def}

 
Die Wahrscheinlichkeiten einer zufälligen Auswahl der Ergebnisse ist für die Referenz-Tests in Tabelle \ref{tab:randomReference} und für das Test-Spiel in Tabelle \ref{tab:randomGame} zu sehen.
In Bezug auf das Test-Spiel gibt es zwei Speller-Matrizen der Größen 5x9 und 5x10.
Zur Vereinfachung wird nur die Wahrscheinlichkeitsverteilung der 5x9-Speller-Matrix zum Vergleich herangezogen, 
da diese die meisten Eingabe-Ermittlungen vorzuweisen hat, darüberhinaus unterscheidet sich die Wahrscheinlichkeitsverteilung der 5x10-Speller-Matrix (dargestellt in Klammern) nur geringfügig von der 5x9-Speller-Matrix.


\begin{table}[h!]
\centering
\footnotesize{
\setlength{\extrarowheight}{5pt}
\begin{tabular}{|ccc|}
\multicolumn{3}{c}{6x6-Speller-Matrix} \\ \hline
\textbf{Hit} & \textbf{Semi} & \textbf{Miss} \\ \hline
2.77\% & 27.77\% & 69.44\% \\ \hline
\end{tabular}}
\caption{Die Wahrscheinlichkeitsverteilung der Referenz-Tests}
\label{tab:randomReference}
\end{table}



\begin{table}[h!]
\centering
\footnotesize{
\setlength{\extrarowheight}{5pt}
\begin{tabular}{|ccc|}
\multicolumn{3}{c}{5x9(5x10)-Speller-Matrix} \\ \hline
\textbf{Hit} & \textbf{Semi} & \textbf{Miss} \\ \hline
2.22\% (2.0\%)& 26.66\% (26.0\%)& 71.11\% (72.0\%) \\ \hline
\end{tabular}}
\caption{Die Wahrscheinlichkeitsverteilung des Test-Spiels}
\label{tab:randomGame}
\end{table}



In Tabelle \ref{tab:Ergebnisse} sind die Testergebnisse der einzelnen Testabschnitte jedes Probanden zu sehen.



\begin{table}[h!]
\centering
\footnotesize{
\setlength{\extrarowheight}{5pt}
\begin{tabular}{cc|ccc|ccc|ccc|}


\multicolumn{2}{c}{}			& \multicolumn{3}{c}{Referenz-Test 1} &	\multicolumn{3}{c}{Test-Spiel} & \multicolumn{3}{c}{Referenz-Test 2} 			\\
\multicolumn{1}{l}{TestNr.}	& \multicolumn{1}{c}{} 			& \multicolumn{1}{c}{\textbf{Hit}}	& \textbf{Semi}	& \multicolumn{1}{c}{\textbf{Miss}}	& \multicolumn{1}{c}{\textbf{Hit}}	& \textbf{Semi}	&\multicolumn{1}{c}{\textbf{Miss}} & \multicolumn{1}{c}{\textbf{Hit}}	& \textbf{Semi}	& \multicolumn{1}{c}{\textbf{Miss}}		\\ \cline{3-11}
\multirow{2}{*}{\textbf{01}}	&	\textbf{\#}	& 6		& 4		& 1		& 6		& 12	& 7		& 7		& 2		& 2		\\
								&	\%			& 54.5\%& 36.4\%& 9.1\%	& 24.0\%& 48.0\%& 28.0\%& 63.6\%& 18.2\%& 18.2\%\\ \cline{3-11}

\multirow{2}{*}{\textbf{02}}	&	\textbf{\#}	& 5		& 5		& 1		& 8		& 12	& 5		& 6		& 4		& 1		\\
								&	\%			& 45.5\%& 45.5\%& 9.1\%	& 32.0\%& 48.0\%& 20.3\%& 54.5\%& 36.4\%	& 9.1\%		\\ \cline{3-11}

\multirow{2}{*}{\textbf{03}}	&	\textbf{\#}	& 8		& 3		& 0		& 4		& 16	& 5		& 7		& 3		& 1		\\
								&	\%			& 72.7\%& 27.3\%& 0.3\%& 16.0\%& 64.0\%& 20.3\%& 63.6\%& 27.3\%& 9.1\% \\ \cline{3-11}

\multirow{2}{*}{\textbf{06}}	&	\textbf{\#}	& 8& 2& 1& 5& 10& 10& 2& 7& 2\\
								&	\%			& 72.7\%& 18.2\%& 9.1\%& 20.3\%& 40.3\%& 40.3\%& 18.2\%& 63.6\%& 18.2\%\\ \cline{3-11}

\multirow{2}{*}{\textbf{08}}	&	\textbf{\#}	& 4& 6& 1& 3& 15& 7& 4& 4& 3\\
								&	\%			& 36.4\%& 54.5\%& 9.1\%& 12.0\%& 60.3\%& 28.0\%& 36.4\%& 36.4\%& 27.3\%\\ \cline{3-11}

\multirow{2}{*}{\textbf{10}}	&	\textbf{\#}	& 6& 3& 2& 1& 11& 13& 7& 3& 1\\
								&	\%			& 54.5\%& 27.3\%& 18.2\%& 4.0\%& 44.0\%& 52.0\%& 63.6\%& 27.3\%& 9.1\%\\ \cline{3-11}	

								
								

%									\\ \cline{3-11}													
% TOTAL
%\multirow{2}{*}{\textbf{Total}}	&	\textbf{\#}	& 37& 23& 6& 27& 76& 47& 33& 23& 10 \\ 
%								&	\textbf{\%} & 56.1\%& 34.8\%& 9.1\%& 18.0\%& 50.7\%& 31.3\%& 50.0\%& 34.8\%& 15.2\% \\ \cline{3-11}
							

\end{tabular}
}
\caption{Die Ergebnisse der Tests}
\label{tab:Ergebnisse}

\end{table}


\pagebreak

In Tabelle \ref{tab:Ergebnisse} ist deutlich zu sehen, dass es sich bei den Eingabe-Ermittlungen der Referenz-Tests und des Test-Spiels nicht um zufällige Ergebnisse handelt,
da diese sich doch erheblich von den Wahrscheinlichkeitsverteilungen in den Tabellen \ref{tab:randomReference} und \ref{tab:randomGame} unterscheiden.

Eine Besonderheit stellen jedoch die vergleichsweise schlechten Ergebnisse von Testperson 10 dar, 
da diese als einzige Erfahrung im Umgang mit \acs{BCI}'s vorweisen konnte und bereits in früheren Stadien der \acs{BCI}-Erweiterung als Testperson fungierte.
Die damaligen Tests wurden jedoch mit einer \textit{StimulusDuration} und \textit{InterStimulusDuration} von jeweils 62.5ms, 
statt der im Test verwendeten 125ms und 125-250ms, durchgeführt und erzielten deutlich bessere Ergebnisse.
Allerdings sind diese Daten unzureichend, um weitere Schlüsse daraus zu ziehen, 
jedoch kann es sinnvoll sein für einzelne \acs{BCI}-Benutzer individuelle \textit{StimulusDuration}- und \mbox{\textit{InterStimulusDuration}}-Werte zu ermitteln, 
um die Genauigkeit weiter zu steigern.\\




\begin{table}[h!]
\centering
\footnotesize{
\begin{tabular}{|p{0.3cm} p{0.5cm}|		>{\columncolor[rgb]{0.84,.95,0.73}}p{0.3cm} >{\columncolor[rgb]{0.84,.95,0.73}}p{0.5cm}| 		p{0.3cm} p{0.5cm}| 		p{0.3cm} p{0.5cm}| 		p{0.3cm} p{0.5cm}| p{0.3cm} p{0.5cm}| p{0.3cm} p{0.5cm}| p{0.3cm} p{0.5cm}| p{0.3cm} p{0.5cm}|}
\hline
\tabbox{1} & \tabbox[b]{\LARGE{1}} &
\tabbox{2} & \tabbox[b]{\LARGE{5}} &
\tabbox{3} & \tabbox[b]{\LARGE{}} &
\tabbox{4} & \tabbox[b]{\LARGE{5}} &
\tabbox{5} & \tabbox[b]{\LARGE{}} &
\tabbox{6} & \tabbox[b]{\LARGE{}} &
\tabbox{7} & \tabbox[b]{\LARGE{2}} &
\tabbox{8} & \tabbox[b]{\LARGE{2}} &
\tabbox{9} & \tabbox[b]{\LARGE{}} \\ \hline

\tabbox{10} & \tabbox[b]{\LARGE{2}} &
\tabbox{11} & \tabbox[b]{\LARGE{6}} &
\tabbox{12} & \tabbox[b]{\LARGE{2}} &
\tabbox{13} & \tabbox[b]{\LARGE{1}} &
\tabbox{14} & \tabbox[b]{\LARGE{1}} &
\tabbox{15} & \tabbox[b]{\LARGE{4}} &
\tabbox{16} & \tabbox[b]{\LARGE{3}} &
\tabbox{17} & \tabbox[b]{\LARGE{}} &
\tabbox{18} & \tabbox[b]{\LARGE{1}} \\ \hline

\rowcolor[rgb]{0.84,.95,0.73}
\tabbox{19} & \tabbox[b]{\LARGE{5}} &
\cellcolor[rgb]{0.57,.81,0.31} \tabbox{20} & \cellcolor[rgb]{0.57,.81,0.31} \tabbox[b]{\LARGE{18}} &
\tabbox{21} & \tabbox[b]{\LARGE{9}} &
\tabbox{22} & \tabbox[b]{\LARGE{8}} &
\tabbox{23} & \tabbox[b]{\LARGE{2}} &
\tabbox{24} & \tabbox[b]{\LARGE{3}} &
\tabbox{25} & \tabbox[b]{\LARGE{3}} &
\tabbox{26} & \tabbox[b]{\LARGE{3}} &
\tabbox{27} & \tabbox[b]{\LARGE{3}} \\ \hline

\tabbox{28} & \tabbox[b]{\LARGE{2}} &
\tabbox{29} & \tabbox[b]{\LARGE{4}} &
\tabbox{30} & \tabbox[b]{\LARGE{1}} &
\tabbox{31} & \tabbox[b]{\LARGE{}} &
\tabbox{32} & \tabbox[b]{\LARGE{}} &
\tabbox{33} & \tabbox[b]{\LARGE{}} &
\tabbox{34} & \tabbox[b]{\LARGE{2}} &
\tabbox{35} & \tabbox[b]{\LARGE{3}} &
\tabbox{36} & \tabbox[b]{\LARGE{1}} \\ \hline

\tabbox{37} & \tabbox[b]{\LARGE{}} &
\tabbox{38} & \tabbox[b]{\LARGE{1}} &
\tabbox{39} & \tabbox[b]{\LARGE{1}} &
\tabbox{40} & \tabbox[b]{\LARGE{}} &
\tabbox{41} & \tabbox[b]{\LARGE{1}} &
\tabbox{42} & \tabbox[b]{\LARGE{}} &
\tabbox{43} & \tabbox[b]{\LARGE{2}} &
\tabbox{44} & \tabbox[b]{\LARGE{}} &
\tabbox{45} & \tabbox[b]{\LARGE{}} \\ \hline

\end{tabular}}

\caption{Das Trefferbild der 5x9-Speller-Matrix des Test-Spiels}
\label{tab:Trefferbild}
\end{table}



In Tabelle \ref{tab:Trefferbild} ist das Trefferbild aller Testpersonen der 5x9-Speller-Matrix dargestellt.
Die \colorbox{hitcolor}{\textit{Hits}} und \colorbox{semicolor}{\textit{Semis}} wurden zur besseren Veranschaulichung farblich hervorgehoben und
lassen offenkundig erkennen, dass sich die Mehrheit der Treffer auf das Ziel\-element und die korrespondierende Zeile bzw. Spalte konzentrieren.\\



Insgesamt erreichte keiner der Tests eine erstrebenswerte Genauigkeit von über 90\%. 
Dies kann allerdings durchaus dem verwendeten \acs{BCI} geschuldet sein, da dieses nicht über die sechs leistungsfähigsten Elektrodenpositionen verfügt.
Diese sechs Elektrodenpositionen würden zusammen eine Genauigkeit von 90\% erreichen \cite[S.220]{wolpaw2012braincomputer} und können mit weiteren Elektroden zusätzlich verbessert und beschleunigt werden.\\


Letztendlich zeigen die Ergebnisse, dass die Aufgabe dieser Masterarbeit zur Erstellung einer \acs{BCI}-Erweiterung für eine Point\&Click Engine erfolgreich realisiert wurde und diese auch funktionsfähig ist.
Ebenfalls ist zu sehen, dass die Ergebnisse des Test-Spiels im Vergleich zu den Referenz-Tests weniger genau sind und eine Verschiebung der Ergebnisse des Test-Spiels zu einer erhöhten Anzahl der Semis zu beobachten ist.\\

Aus diesem Grund werden die Ergebnisse mit Hilfe der Fragebögen und des \acs{NASA-TLX} genauer betrachtet.
Fragebogen 1 und 3 wurden jeweils nach den Referenz-Tests und Fragebogen 2 nach dem Test-Spiel ausgefüllt.\\

\begin{figure}[h!]
\begin{center}
\tikz[scale=1.75] 
\datavisualization[
scientific axes=clean,
y axis={label={Müdigkeit},ticks={step=1},grid}, 
x axis={label={Fragebogen Nr.},ticks={step=1}},
visualize as line/.list={tp01,tp02/08,tp03,tp06,tp10},
style sheet=vary hue,
new legend={below},
% legend={matrix node style={fill=black!5}},
tp01= {label in legend={text=Testperson 01, legend=below}},
tp02/08= {label in legend={text=Testperson 02/08, legend=below}},
tp03= {label in legend={text=Testperson 03, legend=below}},
tp06= {label in legend={text=Testperson 06, legend=below}},
tp10= {label in legend={text=Testperson 10, legend=below}},
]

data [set=tp01] {
x, y
0, 2
1, 2
2, 3
3, 3
}

data [set=tp02/08] {
x, y
0, 3
1, 3
2, 3
3, 3
}

data [set=tp03] {
x, y
0, 7
1, 5
2, 8
3, 8
}

data [set=tp06] {
x, y
0, 3
1, 3
2, 4
3, 5
}

data [set=tp10] {
x, y
0, 3
1, 4
2, 4
3, 4
};

\caption{Müdigkeitsverlauf aller Testpersonen}
\label{muedigkeit}
\end{center}
\end{figure}

In Abbildung \ref{muedigkeit} ist der Müdigkeitsverlauf aller Testpersonen zu sehen. 
Bis auf Testperson 02 gaben alle eine erhöhte Müdigkeit gegen Ende der Tests an.
Insbesondere fällt auf, dass Testperson 02, die keine Ermüdung aufwies, auch die besten Ergebnisse des Test-Spiels vorweisen konnte.
Gleiches gilt für den Verlauf des allgemeinen Wohlbefindens in Abbildung \ref{wohlbefinden}, bis auf Testperson 02 gaben alle Probanden einen verschlechterten Allgemeinzustand an.\\

\pagebreak


\begin{figure}[h!]
\begin{center}
\tikz[scale=1.75] 
\datavisualization[
scientific axes=clean,
y axis={label={Wohlbefinden},ticks={step=1},grid}, 
x axis={label={Fragebogen Nr.},ticks={step=1}},
visualize as line/.list={tp01,tp02,tp03,tp06,tp08,tp10},
style sheet=vary hue,
new legend={below},
% legend={matrix node style={fill=black!5}},
tp01= {label in legend={text=Testperson 01, legend=below}},
tp02= {label in legend={text=Testperson 02, legend=below}},
tp03= {label in legend={text=Testperson 03, legend=below}},
tp06= {label in legend={text=Testperson 06, legend=below}},
tp08= {label in legend={text=Testperson 08, legend=below}},
tp10= {label in legend={text=Testperson 10, legend=below}},
]

data [set=tp01] {
x, y
0, 9
1, 8
2, 7
3, 8
}

data [set=tp02] {
x, y
0, 9
1, 9
2, 9
3, 9
}

data [set=tp03] {
x, y
0, 6
1, 5
2, 5
3, 4
}

data [set=tp06] {
x, y
0, 5
1, 5
2, 5
3, 3
}

data [set=tp08] {
x, y
0, 7
1, 7
2, 5
3, 5
}

data [set=tp10] {
x, y
0, 7
1, 7
2, 6
3, 6
};

\caption{Der Verlauf des allgemeinen Wohlbefindens aller Testpersonen}
\label{wohlbefinden}
\end{center}
\end{figure}

Die Ergebnisse in Abbildung \ref{muedigkeit} und \ref{wohlbefinden} geben allerdings wenig Aufschluss darüber, 
warum sich die Ergebnisse des Test-Spiels von den Ergebnissen der Referenz-Tests unterscheiden.
Demgegenüber zeigt Abbildung \ref{konzentration}, das einige der Testpersonen es als schwerer empfanden sich auf die Matrixelemente des Test-Spiels zu konzentrieren.\\

\begin{figure}[h!]
\begin{center}
\tikz[scale=1.75] 
\datavisualization[
scientific axes=clean,
y axis={label={Konzentrationsschwierigkeit},ticks={step=1},grid}, 
x axis={label={Fragebogen Nr.},ticks={step=1}},
visualize as line/.list={tp01,tp02,tp03,tp06,tp08,tp10},
style sheet=vary hue,
new legend={below},
% legend={matrix node style={fill=black!5}},
tp01= {label in legend={text=Testperson 01, legend=below}},
tp02= {label in legend={text=Testperson 02, legend=below}},
tp03= {label in legend={text=Testperson 03, legend=below}},
tp06= {label in legend={text=Testperson 06, legend=below}},
tp08= {label in legend={text=Testperson 08, legend=below}},
tp10= {label in legend={text=Testperson 10, legend=below}},
]

data [set=tp01] {
x, y
1, 4
2, 5
3, 3
}

data [set=tp02] {
x, y
1, 3
2, 2
3, 2
}

data [set=tp03] {
x, y
1, 4
2, 7
3, 6
}

data [set=tp06] {
x, y
1, 3
2, 5
3, 4
}

data [set=tp08] {
x, y
1, 3
2, 5
3, 6
}

data [set=tp10] {
x, y
1, 5
2, 5
3, 3
};

\caption{Das Schwierigkeitsempfinden der Testpersonen, sich auf die jeweiligen Matrixelemente zu fokussieren.}
\label{konzentration}
\end{center}
\end{figure}

Insbesondere weisen Anmerkungen einiger Testpersonen darauf hin, dass es ihnen schwerer gefallen ist, sich auf die im Vergleich zum Referenz-Test kleineren Matrixelemente zu konzentrieren.
Weiterhin wurde angegeben, dass benachbarte Spalten oder Zeilen mehr wahrgenommen wurden, als bei den Referenz-Tests.\\
Die kleineren Stimuli des Test-Spiels sind demnach wahrscheinlich für die Diskrepanz zwischen dem Test-Spiel und den Referenz-Tests verantwortlich.
Allerdings handelt es sich beim Test-Spiel und bei Point\&Click-Spielen im allgemeinen nicht um sterile Hintergründe, sodass auch andere Faktoren eine Rolle bei der Diskrepanz der Ergebnisse spielen können.\\






% NASA-TLX
\begin{table}[h!]
%\centering
\footnotesize{
\setlength{\extrarowheight}{5pt}
\begin{tabular}{
|c|cc|cc|cc|cc|cc|cc|
}


\multicolumn{1}{c}{} 
& \multicolumn{2}{c}{Test 01} 
& \multicolumn{2}{c}{Test 02} 
& \multicolumn{2}{c}{Test 03} 
& \multicolumn{2}{c}{Test 06} 
& \multicolumn{2}{c}{Test 08} 
& \multicolumn{2}{c}{Test 10} 
\\


\multicolumn{1}{r|}{\tiny{W=Wertung, G=Gewichtung}} 
& W & G
& W & G
& W & G
& W & G
& W & G
& W & G
\\ \hline


Geistige Anforderung
& 0.5 & 5
& 0.85 & 5
& 0.5 & 1
& 0.5 & 4
& 0.85 & 4
& 0.9 & 5
\\

Körperliche Anforderung
& 0.65 & 1
& 0.05 & 0
& 0.5 & 0
& 0.00 & 1
& 0.65 & 3
& 0.1 & 1
\\

Zeitliche Anforderung
& 0.7 & 3
& 0.55 & 3
& 0.05 & 3
& 0.05 & 0
& 0.15 & 0
& 0.5 & 0
\\

Ausführung der Aufgaben
& 0.55 & 2
& 0.25 & 3
& 0.3 & 3
& 0.5 & 2
& 0.5 & 5
& 0.25 & 2
\\

Anstrengung
& 0.5 & 0
& 0.25 & 2
& 0.7 & 4
& 0.5 & 3
& 0.85 & 3
& 0.9 & 3
\\

Frustration
& 0.6 & 4
& 0.25 & 2
& 0.8 & 4
& 0.7 & 5
& 0.3 & 0
& 0.9 & 4
\\ \hline \hline

Subjektive Belastung
& \multicolumn{2}{c|}{0.583}
& \multicolumn{2}{c|}{0.51}
& \multicolumn{2}{c|}{0.503}
& \multicolumn{2}{c|}{0.533}
& \multicolumn{2}{c|}{0.693}
& \multicolumn{2}{c|}{0.76}
\\ \hline



\end{tabular}
}
\caption{Die Ergebnisse des \acs{NASA-TLX}}
\label{tab:nasatlxresults}

\end{table}







Abschließend werden noch die Ergebnisse des \acs{NASA-TLX} betrachtet.
In Tabelle \ref{tab:nasatlxresults} wird die subjektive Belastung der Probanden dargestellt.
Testperson 10 weist bei diesem Test die höchste subjektive Belastung auf und könnte eine Erklärung für ihr vergleichweise schlechtes Ergebnis beim Test-Spiel sein.
Zusätzlich weist auch Testperson 08 eine ähnliche hohe subjektive Belastung und die zweit schlechtesten Ergebnisse auf.
Dies führt zu der Annahme, dass die Ergebnisse stark von der subjektiven Belastung beeinflusst werden können und eine entsprechende Erklärung für individuelle schlechte Ergebnisse sein kann.\\

Hinzu kommt, dass Testperson 02, 06 und 08 Brillenträger waren und es auf Grund des \acs{BCI} nicht möglich war die Brillen während der Tests zu tragen.
Testperson 06 und 10 gaben zudem Schlafstörungen als neurologische Erkrankungen an.
In Bezug auf diese Eigenschaften kann in den Testergebnissen keine klare Tendenz gesehen werden,
dennoch ist es anzunehmen, dass diese Faktoren Einfluss auf individuelle Ergebnisse haben können.\\











