\chapter{Fazit und Ausblick}
Zum Ende werden Folgerungen und Rückschlüsse aus dem Verlauf und den Ergebnissen dieser Arbeit gezogen.
Etwaige Besonderheiten oder Probleme die während der Bearbeitung auftraten werden ebenso in die Bewertung miteinfließen.
Im Anschluss wird zudem ein Ausblick über mögliche Entwicklungs- und Verbesserungsmöglichkeiten gegeben.\\


\section{Fazit}

Ziel der vorliegenden Arbeit war es eine \acs{BCI}-Erweiterung für eine Point\&Click \mbox{Engine} zu erstellen.
Zu diesem Zweck wurden geeignete Softwarekomponenten ausgewählt und darauf basierend eine Erweiterung konzipiert,
die dem Eingabeparadigma der Point\&Click Engine gerecht wurde.
Während des Konzepts wurden daher mögliche Anwendungsszenarien vorgestellt, bewertet und entsprechend ausgewählt,
so dass die \acs{BCI}-Erweiterung die Anforderungen dieser Masterarbeit erfüllt.
Im Zuge dessen ergab sich auf Grund zeitkritischer Komponenten des P300-Spellers, dass die Korrektheit der Erweiterung erst nach abschließender Tests verifiziert werden konnte.\\

Die Ermittlung der Testparameter stellte sich indes als deutlich zeitaufwändiger heraus als erwartet.
Dies war in erster Linie der Tatsache geschuldet, dass Selbst-Tests weder mit der \acs{BCI}-Erweiterung, noch mit dem P300-Speller des BCI2000-Frameworks funktionierten,
so dass zur Ermittlung der Testparameter auf Testpersonen zurückgegriffen werden musste.
Zudem zeigte sich, dass das verwendete Emotiv EPOC sich als nicht optimal in Bezug auf das P300 Paradigma erwies, da es nicht über die Elektrodenpositionen mit größter Ausprägung des \acs{P300 ERP} verfügt 
und für die Genauigkeit ein allgemeines Problem darstellte.
Dies hatte zur folge, dass die Anzahl der Stimuli-Sequenzen erhöht werden musste, um überhaupt brauchbare Ergebnisse zu erzielen.
Insgesamt benötigte daher jeder Test mehr als eine Stunde und erschwerte die Suche nach freiwilligen Testpersonen.
Letztendlich konnten jedoch genug Testergebnisse gesammelt 
und wie in den Testergebnissen gezeigt wurde, konnte die erwartete Funktion verifiziert werden, da die beobachteten Ergebnisse nicht zufälligen Ursprungs waren.
In Bezug auf die Genauigkeit, ist es allerdings sinnvoll weitere Tests mit einem präziseren \acs{BCI} durchzuführen.
Zudem brachten die Tests die Erkenntnis, dass die Größe und Komplexität der Speller-Matrix eine große Rolle bei der Genauigkeit spielt.
Dies kann genutzt werden, um eine optimale Größe der Matrixelemente zu bestimmen.
Darüberhinaus sind individuelle Eigenschaften wie Konzentrationsfähigkeit, Müdigkeit und Disziplin wichtige Faktoren, die die Genauigkeit ebenso maßgeblich beeinflussen können.
Insgesamt waren jedenfalls fast alle Nutzer in der Lage, trotz der geringen Präzision, die Erweiterung zu steuern.\\
Abschließend lässt sich hieraus der Schluss ziehen, dass die Aufgabenstellung dieser Masterarbeit erfolgreich bearbeitet wurde.\\









\pagebreak
\section{Ausblick}

Die Erweiterung ermöglicht eine Vielzahl an zukünftigen Entwicklungen. 
Zum einen können Point\&Click Spiele für Patienten entwickelt werden, die diese tatsächlich spielen und somit von dieser Masterarbeit profitieren. 
Die Entwicklung eines Point\&Click Spiels unter Verwendung der \acs{BCI}-Erweiterung stellt dabei keine Herausforderung dar und könnte von unerfahrenen Personen durchgeführt werden.
Zum anderen können ebenso weitere Test-Spiele erstellt werden, so dass beispielsweise optimale Parameter für \textit{StimulusDuration} und \textit{InterStimulusDuration} ermittelt werden.\\

Weiterhin ist es möglich das Konzept dieser Arbeit auf andere Spiele Engine's zu übertragen, 
dies würde lediglich eine Modifikation der Engine erfordern und könnte auf die gleiche Weise mit dem modifzierten P300 Speller kommunizieren.
Natürlich ist es ebenso möglich das Konzept weiter zu entwickeln, sodass die Logik sich ausschließlich innerhalb der Erweiterung auf Engine-Ebene befindet.\\

Prinzipiell ist es sogar möglich, mit Hilfe ähnlicher Bild- und Matrix-Parameter, 
viele verschiedene Anwendungen so zu erweitern, 
dass diese ebenfalls über den modifizierten P300-Speller dieser Arbeit gesteuert werden können.
Insofern ist zu hoffen, dass in Zukunft noch weitere Anwendungen für \acs{BCI}s entwickelt oder erweitert werden und motorisch eingeschränkte Menschen von diesen Entwicklungen profitieren können.\\







